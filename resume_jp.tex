\documentclass[letterpaper,11pt]{article}

\usepackage{latexsym}
\usepackage[empty]{fullpage}
\usepackage{titlesec}
\usepackage{marvosym}
\usepackage[usenames,dvipsnames]{color}
\usepackage{verbatim}
\usepackage{enumitem}
\usepackage[hidelinks]{hyperref}
\usepackage{fancyhdr}
\usepackage[english]{babel}
\usepackage{tabularx}

\pagestyle{fancy}
\fancyhf{} % clear all header and footer fields
\fancyfoot{}
\renewcommand{\headrulewidth}{0pt}
\renewcommand{\footrulewidth}{0pt}

% Adjust margins
\addtolength{\oddsidemargin}{-0.5in}
\addtolength{\evensidemargin}{-0.5in}
\addtolength{\textwidth}{1in}
\addtolength{\topmargin}{-.5in}
\addtolength{\textheight}{1.0in}

\urlstyle{same}

\raggedbottom
\raggedright
\setlength{\tabcolsep}{0in}

% Sections formatting
\titleformat{\section}{
  \vspace{-4pt}\scshape\raggedright\large
}{}{0em}{}[\color{black}\titlerule \vspace{-5pt}]

% Custom commands
\newcommand{\resumeItem}[1]{
  \item\large{
    {#1 \vspace{0pt}}
  }
}

\newcommand{\resumeSubheading}[4]{
  \vspace{-2pt}\item
    \begin{tabular*}{0.97\textwidth}[t]{l@{\extracolsep{\fill}}r}
      \textbf{#1} & #2 \\
      \textit{\small#3} & \textit{\small #4} \\
    \end{tabular*}\vspace{-7pt}
}

\newcommand{\resumeSubSubheading}[2]{
    \item
    \begin{tabular*}{0.97\textwidth}{l@{\extracolsep{\fill}}r}
      \textit{\small#1} & \textit{\small #2} \\
    \end{tabular*}\vspace{-7pt}
}

\newcommand{\resumeProjectHeading}[2]{
    \item
    \begin{tabular*}{0.97\textwidth}{l@{\extracolsep{\fill}}r}
      \small#1 & #2 \\
    \end{tabular*}\vspace{-7pt}
}

\newcommand{\resumeSubItem}[1]{\resumeItem{#1}\vspace{-4pt}}

\renewcommand\labelitemii{$\vcenter{\hbox{\tiny$\bullet$}}$}

\newcommand{\resumeSubHeadingListStart}{\begin{itemize}[leftmargin=0.15in, label={}]}
\newcommand{\resumeSubHeadingListEnd}{\end{itemize}}
\newcommand{\resumeItemListStart}{\begin{itemize}}
\newcommand{\resumeItemListEnd}{\end{itemize}\vspace{-5pt}}


\usepackage{array}

% Define a custom column type
\newcolumntype{L}[1]{>{\raggedright\arraybackslash}p{#1}} % left-aligned column with fixed width
\newcolumntype{R}{>{\raggedright\arraybackslash}X}       % right-aligned column, auto width

%%%%%%  RESUME STARTS HERE  %%%%%%%%%%%%%%%%%%%%%%%%%%%%

\begin{document}

%----------HEADING----------
\begin{center}
    \textbf{\Huge \scshape 近藤 昌史} \\ \vspace{1pt}
    \textbf{\Large \scshape シニアソフトウェアエンジニア} \\
    (000)-000-0000 $|$ Vancouver, British Columbia, Canada \\
    \href{https://linkedin.com/in/masashi-kondo}{\underline{linkedin.com/in/masashi-kondo}} $|$
    \href{https://github.com/MssKnd}{\underline{github.com/MssKnd}}
\end{center}

%-----------SUMMARY-----------
\section{概要}
ソフトウェアエンジニアとして10年以上の経験を持ち、特にWebフロントエンドを専門としています。技術的な課題を解決し、開発のニーズがビジネス目標と合致するように調整します。TypeScriptとReactを得意とし、バックエンド開発、Web標準API、GitHub Actionsを活用したCI/CDにも精通しています。チームのコミュニケーションと生産性を向上させ、チームの効率を高めるリーダーシップを発揮します。GDG DevFest 2020 Tokyoで登壇経験あり。

%-----------EXPERIENCE-----------
\section{職務経験}
  \resumeSubHeadingListStart

    \resumeSubheading
      {CEO / ソフトウェアエンジニア(個人事業主)}{2022年06月 -- 現在}
      {合同会社 Rokka}{熊本、日本}
      \resumeItemListStart
        \resumeItem{スタートアップ向けの SaaS 製品を開発し、高凝集・低結合のシステム設計に注力。}
        \resumeItem{UI/UX の一貫性と DX のためのデザインシステムを設計・実装。}
        \resumeItem{Web フロントエンドアプリケーションで、大量データの表示・操作に関するパフォーマンス問題を改善。メモリ効率を向上させ、レンダリング時間を 1 分から 5 秒に短縮。}
        \resumeItem{スクラムマスターとして、スクラムイベントの効率を向上。会議の生産性を2倍にし、フィードバックサイクルを改善することで、チーム全体の生産性を 20\% 向上させた。}
      \resumeItemListEnd

    \resumeSubheading
      {ソフトウェアエンジニア / Web フロントエンドエンジニア}{2018年12月 -- 2022年03月}
      {セーフィー株式会社}{東京、日本}
      \resumeItemListStart
        \resumeItem{IoT カメラサービスのための録画映像ビューアを Angular で開発}
        \resumeItem{ネットワークカメラの映像に機械学習を統合し、画像アノテーションのためのキャンバス描画アプリケーションを開発}
        \resumeItem{開発チームが 2 人から 6 人に成長する中で、スクラムの導入を主導。ベロシティのトラッキングと可視化を行い、開発プロセスを安定化。}
      \resumeItemListEnd

    \resumeSubheading
      {ネットワークエンジニア}{2013年04月 -- 2018年12月}
      {ソフトバンク株式会社}{東京、日本}
      \resumeItemListStart
        \resumeItem{専用線システムの保守およびWebアプリケーションの開発を担当}
        \resumeItem{国際拠点間の VPN を利用した業務システムを2ヶ月で設計・導入}
      \resumeItemListEnd

  \resumeSubHeadingListEnd

%-----------EDUCATION-----------
\section{学歴}
  \resumeSubHeadingListStart
    \resumeSubheading
      {感性科学 修士課程}{2011年04月 -- 2013年03月}
      {九州大学大学院}{}
    \resumeSubheading
      {情報科学(コンピュータサイエンス相当)学士課程}{2009年04月 -- 2011年03月}
      {熊本高等専門学校}{}
  \resumeSubHeadingListEnd
  
%-----------SKILLS-----------
\section{スキル}
  \setlength{\leftskip}{0.15in}
  \begin{tabularx}{\dimexpr\textwidth-0.15in\relax}{L{2.18in} R}
    \textbf{開発言語:} & TypeScript, Deno, JavaScript, Kotlin, HTML/CSS \\
    \textbf{フレームワーク・ライブラリ:} & React, Angular, Vue.js, Next.js, jest, Storybook, Playwright, Spring boot \\
    \textbf{ツール・プラットフォーム:} & Git, GitHub Actions, Node.js, Supabase, PostgreSQL, Figma, Docker, AWS \\
    \textbf{その他:} & Web Standard API, DDD, Secure cording, Agile methodologies, accessibility standards (WCAG), Network Security
  \end{tabularx}

%-----------CERTIFICATIONS-----------
\section{Certifications}
\begin{tabularx}{\textwidth}{L{4in}@{\hskip 1em}R}
  \textbf{・ AWS Certified Solutions Architect - Associate} & \textbf{・ ネットワークスペシャリスト} \\
  \textbf{・ 応用情報技術者} & \textbf{・ ITIL FOUNDATION V3}
\end{tabularx}

\end{document}